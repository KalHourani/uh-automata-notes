\subsection{Regular Languages}\label{subsec:regular-languages}
\subsubsection{Finite Automata}

\begin{definition}
A \textbf{Deterministic Finite-State Automata} (DFA) or \textbf{Finite-State Machine} is a quintuple $(A, Q, \tau, q_0, \mathcal{F})$ where 

\begin{align*}
    &A\text{ is the }\textbf{alphabet}\\
    &Q\text{ is a finite, non-empty }\textbf{set of states}\\
    &\tau:Q\times A\to Q\text{ is the }\textbf{transition function }\\
    &q_0\text{ is the }\textbf{initial state}\\
    &\mathcal{F}\subseteq Q\text{ is the set of }\textbf{final states}
\end{align*}
\end{definition}
We can extend $\tau$ as follows: 

$\tau^*:Q\times A^*\to Q$

\[\tau^*(q, s) = \begin{cases} q &\mbox{ if } s=\varepsilon\\
                             \tau^*(\tau(q, s_0), s') &\mbox{ if } s=s_0\cdot s'\end{cases}\]

We proceed informally and use $\tau$ to refer to $\tau^*$. 

Consider the following DFA:

\begin{center}\begin{tikzpicture}[shorten >=1pt,node distance=2cm,on grid,auto]
  \tikzstyle{every state}=[fill={rgb:black,1;white,10}]

  \node[state,initial](q_0){$q_0$};
  \node[state, accepting](q_1)[right of=q_0]{$q_1$};
  \node[state, accepting](q_2)[below of=q_1]{$q_2$};
  \node[state](q_3)[left of=q_2]{$q_3$};

  \path[->]
  (q_0) edge                node {$a$}  (q_1)
        edge  [loop above]  node {$b$}  ()
  (q_1) edge                node {$a$}  (q_2)
        edge  [loop above]  node {$b$}  ()
  (q_2) edge                node {$a$}  (q_3)
        edge  [loop below]  node {$b$}  ()
  (q_3) edge                node {$a$}  (q_0)
        edge  [loop below]  node {$b$}  ();
\end{tikzpicture}\end{center}

The figure indicates that we begin at state $q_0$. The double-circles for states $q_1$ and $q_2$ indicate that they are accepting or final states. An arrow indicates the state to move to after receiving an input. For example, if we receive the input string $abba$, we begin at state $q_0$ and receive $a$, so we move to state $q_1$. We then receive $b$ and stay in $q_1$. We repeat this for the next symbol, $b$, and then move to $q_2$ upon receiving the final $a$. Since $q_2$ is a final state, we say that this DFA \textbf{accepts} the string $abba$. 

We can represent the above DFA using a table, as follows:

\begin{center}\begin{tabular}{r|c c r}
         & $a$ & $b$ & \\\hline
    $\to q_0$ & $q_1$ & $q_0$ & 0 \\
    $q_1$ & $q_2$ & $q_1$ & 1 \\
    $q_2$ & $q_3$ & $q_2$ & 1\\
    $q_3$ & $q_0$ & $q_3$ & 0\\
\end{tabular}\end{center}

The first column indicates the states, while the first row indicates the symbols. The final column indicates whether a state is accepting: 0 refers to a non-final state, 1 to a final state. The remaining values indicate the transition function $\tau$, e.g. $\tau(q_0, a)=q_1$, indicated by the entry corresponding to row $q_0$ and column $a$. Finally, the arrow pointing to $q_0$ indicates that it is the starting position. 

\begin{definition}
Let $\underset{\sim}{D}$ be some DFA. Then $L(\underset{\sim}{D})$, the language accepted by the DFA, is \[\{s\in A^*|\tau(q_0, s)\in \mathcal{F}\}\]
\end{definition}

\begin{definition}
A language is \textbf{regular} if and only if there exists a DFA that accepts it.
\end{definition}

\begin{definition}
A \textbf{Non-Deterministic Finite-State Automata} (NFA) is a quintuple $(A, Q, \tau, q_0, \mathcal{F})$ where 
\end{definition}

\begin{align*}
    &A\text{ is the }\textbf{alphabet}\\
    &Q\text{ is a finite, non-empty }\textbf{set of states}\\
    &\tau:Q\times A\to 2^Q\text{ is the }\textbf{transition function }\\
    &q_0\text{ is the }\textbf{initial state}\\
    &\mathcal{F}\subseteq Q\text{ is the set of }\textbf{final states}
\end{align*}

We can extend $\tau$ as follows: 

$\tau^*:2^Q\times A^*\to 2^Q$

\[\tau^*(P, s) = \begin{cases} P &\mbox{ if } s=\varepsilon\\
                             \tau^*\left(\displaystyle\bigcup_{q\in P}\tau(q, s_0), s'\right) &\mbox{ if } s=s_0\cdot s'\end{cases}\]

We proceed informally and use $\tau$ to refer to $\tau^*$. 

Consider the following NFA:

\begin{center}\begin{tikzpicture}[shorten >=1pt,node distance=2cm,on grid,auto]
  \tikzstyle{every state}=[fill={rgb:black,1;white,10}]

  \node[state, initial](q_0){$q_0$};
  \node[state, accepting](q_1)[right of=q_0]{$q_1$};
  \node[state, accepting](q_2)[below of=q_1]{$q_2$};
  \node[state](q_3)[left of=q_2]{$q_3$};

  \path[->]
  (q_0) edge                node {$a$}  (q_1)
        edge  [loop above]  node {$b$}  ()
  (q_1) edge                node {$a$}  (q_2)
        edge                node {$a$}  (q_3)
  (q_2) edge                node {$a$}  (q_3)
        edge  [loop below]  node {$b$}  ()
  (q_3) edge                node {$a$}  (q_0)
        edge  [loop below]  node {$b$}  ();
\end{tikzpicture}\end{center}

The diagrams for an NFA and DFA follow the same notation. However, the notation for the table differs slightly:

\begin{center}\begin{tabular}{r|c c r}
         & $a$ & $b$ & \\\hline
    $\to q_0$ & $q_1$ & $q_0$ & 0 \\
    $q_1$ & $q_2q_3$ & $\emptyset$ & 1 \\
    $q_2$ & $q_3$ & $q_2$ & 1\\
    $q_3$ & $q_0$ & $q_3$ & 0\\
\end{tabular}\end{center}

The values of the transition function are now sets. We informally refer to the set $\{q_0\}$ by $q_0$, and similarly the set $\{q_2, q_3\}$ by $q_2q_3$. In some cases, to avoid ambiguity, we will use commas, e.g. we may represent $\{q_2, q_3\}$ as $q_2,q_3$. We similarly say, given a string $s$, if there exists a path through an NFA that ends in a final state, we say that the NFA \textbf{accepts} $s$.

Similarly, we define the set of languages accepted by an NFA $\underset{\sim}{N}$, $L(\underset{\sim}{N})$, as \[L(\underset{\sim}{N}) = \{s\in A^*|\tau(q_0, s)\cap\mathcal{F}\neq\emptyset\}\]

It should be clear that each DFA is an NFA, but the reverse is not true. However, we can convert an NFA to a DFA on the powerset $2^{\mathcal{Q}}$ by using the \textbf{subset construction}: begin with the initial state and traverse the NFA, adding unseen states to the left-most column until all paths have been exhausted. For example, with our NFA above, we begin with:

\begin{center}\begin{tabular}{r|c c r}
         & $a$ & $b$ & \\\hline
    $\to q_0$ & $q_1$ & $q_0$ & 0 \\
\end{tabular}\end{center}

$q_0$ has already been seen, so we ignore it. $q_1$ is new, so we add it to the table:

\begin{center}\begin{tabular}{r|c c r}
         & $a$ & $b$ & \\\hline
    $\to q_0$ & $q_1$ & $q_0$ &  \\
          $q_1$ &       &       & 
\end{tabular}\end{center}

We now visit the corresponding states of $q_1$, which are $q_2q_3$ and $\emptyset$, both of which have not yet been visited. 

\begin{center}\begin{tabular}{r|c c r}
         & $a$ & $b$ & \\\hline
    $\to q_0$ & $q_1$ & $q_0$ &  \\
          $q_1$ & $q_2q_3$ & $\emptyset$ & \\
          $q_2q_3$ & & & \\
          $\emptyset$ & & &
\end{tabular}\end{center}

When $q_2$ receives $a$, it transitions to state $q_3$. When $q_3$ receives $a$, it transitions to state $q_0$, so $q_2q_3$ transitions to $q_0q_3$. Similarly, $q_2q_3$ transitions to state $q_2q_3$ when it receives $b$.

\begin{center}\begin{tabular}{r|c c r}
         & $a$ & $b$ & \\\hline
    $\to q_0$ & $q_1$ & $q_0$ &  \\
          $q_1$ & $q_2q_3$ & $\emptyset$ & \\
          $q_2q_3$ & $q_0q_3$ & $q_2q_3$ & \\
          $\emptyset$ & & &
\end{tabular}\end{center}

The empty set transitions to the empty set, by definition. 

\begin{center}\begin{tabular}{r|c c r}
         & $a$ & $b$ & \\\hline
    $\to q_0$ & $q_1$ & $q_0$ &  \\
          $q_1$ & $q_2q_3$ & $\emptyset$ & \\
          $q_2q_3$ & $q_0q_3$ & $q_2q_3$ & \\
          $\emptyset$ & $\emptyset$ & $\emptyset$ &
\end{tabular}\end{center}

$q_0q_3$ has not yet been visited, so we add it to the left-most column:

\begin{center}\begin{tabular}{r|c c r}
         & $a$ & $b$ & \\\hline
    $\to q_0$ & $q_1$ & $q_0$ &  \\
          $q_1$ & $q_2q_3$ & $\emptyset$ & \\
          $q_2q_3$ & $q_0q_3$ & $q_2q_3$ & \\
          $\emptyset$ & $\emptyset$ & $\emptyset$ &\\
          $q_0q_3$ & & & 
\end{tabular}\end{center}

Then we visit its corresponding states:

\begin{center}\begin{tabular}{r|c c r}
         & $a$ & $b$ & \\\hline
    $\to q_0$ & $q_1$ & $q_0$ &  \\
          $q_1$ & $q_2q_3$ & $\emptyset$ & \\
          $q_2q_3$ & $q_0q_3$ & $q_2q_3$ & \\
          $\emptyset$ & $\emptyset$ & $\emptyset$ &\\
          $q_0q_3$ & $q_0q_1$ & $q_0q_3$ & 
\end{tabular}\end{center}

Continuing, we end with the following DFA:

\begin{center}\begin{tabular}{r|c c r}
         & $a$ & $b$ & \\\hline
    $\to q_0$ & $q_1$ & $q_0$ &  \\
          $q_1$ & $q_2q_3$ & $\emptyset$ & \\
          $q_2q_3$ & $q_0q_3$ & $q_2q_3$ & \\
          $\emptyset$ & $\emptyset$ & $\emptyset$ &\\
          $q_0q_3$ & $q_0q_1$ & $q_0q_3$ & \\
          $q_0q_1$ & $q_1q_2q_3$ & $q_0$ & \\
          $q_1q_2q_3$ & $q_0q_2q_3$ & $q_2q_3$ & \\
          $q_0q_2q_3$ & $q_0q_1q_3$ & $q_0q_2q_3$ & \\
          $q_0q_1q_3$ & $q_0q_1q_2q_3$ & $q_0q_3$ & \\
          $q_0q_1q_2q_3$ & $q_0q_1q_2q_3$ & $q_0q_2q_3$ & \\
\end{tabular}\end{center}

However, we need to include the accepting states. The accepting states of the NFA are $q_1$ and $q_2$, and thus any state including either state is accepting:

\begin{center}\begin{tabular}{r|c c r}
         & $a$ & $b$ & \\\hline
    $\to q_0$ & $q_1$ & $q_0$ & 0 \\
          $q_1$ & $q_2q_3$ & $\emptyset$ & 1 \\
          $q_2q_3$ & $q_0q_3$ & $q_2q_3$ & 1 \\
          $\emptyset$ & $\emptyset$ & $\emptyset$ & 0\\
          $q_0q_3$ & $q_0q_1$ & $q_0q_3$ & 0 \\
          $q_0q_1$ & $q_1q_2q_3$ & $q_0$ & 1 \\
          $q_1q_2q_3$ & $q_0q_2q_3$ & $q_2q_3$ & 1 \\
          $q_0q_2q_3$ & $q_0q_1q_3$ & $q_0q_2q_3$ & 1 \\
          $q_0q_1q_3$ & $q_0q_1q_2q_3$ & $q_0q_3$ & 1 \\
          $q_0q_1q_2q_3$ & $q_0q_1q_2q_3$ & $q_0q_2q_3$ & 1\\
\end{tabular}\end{center}

Note that an NFA does not necessarily admit a DFA with as many states. Consider the following example:

\begin{center}\begin{tabular}{r| c c r}
     & $a$ & $b$ & \\\hline
$\to 0$&$\{1,2,\hdots,n\}$ & 0 & 0 \\
     1 & 2 & 1 & 0 \\
     2 & 3 & 2 & 0 \\
     \vdots & \vdots & \vdots & \vdots \\
     $i$ & $i + 1$ & $i$ & 0 \\
     \vdots & \vdots & \vdots & \vdots \\
     $n-1$ & $n$ & $n-1$ & 0 \\
     $n$ & 1 & $n$ & 1
\end{tabular}\end{center}

The NFA above admits the following DFA:

\begin{center}\begin{tabular}{r| c c r}
     & $a$ & $b$ & \\\hline
$\to 0$&$\{1,2,\hdots,n\}$ & 0 & 0 \\
     $\{1, 2,\hdots, n\}$ & $\{1, 2,\hdots, n\}$ & $\{1, 2,\hdots, n\}$ & 1 \\
\end{tabular}\end{center}

The above DFA contains only 2 states, despite the NFA containing $n+1$ states.

That every NFA admits a DFA which accepts the same language shows that the class of languages denoted by DFAs, $\mathcal{L}_{\text{DFA}}$, is the same as the class of languages denoted by NFAs, $\mathcal{L}_{\text{NFA}}$, i.e, that

\[\mathcal{L}_{\text{DFA}} = \mathcal{L}_{\text{NFA}}\]

For an NFA, there is no guarantee of a unique smallest NFA which accepts the same strings. However, for a DFA, such a notion exists.

Consider two states, $p$ and $q$, and corresponding $L_p$ and $L_q$, where $L_p$ has initial state $p$ and $L_q$ has initial state $q$. We say that $p$ and $q$ are distinguishable if there exists a string $s$ such that $s$ is in $L_p$ and not in $L_q$, or vice-versa. We use this notion to \textbf{reduce} a DFA.

Begin with a partition of $Q$ into subsets $\mathcal{F}$ and $Q-\mathcal{F}$, i.e., the accepting and rejecting states. For a pair of states $p, q$ if the result of transitioning $p$ and $q$ falls into different partitions, we partition the subset and continue.

For example, given the following DFA:

\begin{center}\begin{tabular}{r| c c r}
         & $a$ & $b$ & \\\hline
         $\to 0$ & 1 & 2 & 0\\
               1 & 2 & 3 & 1\\
               2 & 3 & 4 & 0\\
               3 & 0 & 5 & 1\\
               4 & 5 & 6 & 0\\
               5 & 6 & 7 & 1\\
               6 & 7 & 0 & 0\\
               7 & 4 & 1 & 1
    \end{tabular}\end{center}

We have two partitions:

\begin{center}\begin{tabular}{c|c|c|c|c|c|c|c}
\multicolumn{4}{c|}{Rejecting} & \multicolumn{4}{|c}{Accepting}\\\hline
\multicolumn{4}{c|}{0, 2, 4, 6} & \multicolumn{4}{|c}{1, 3, 5, 7}
\end{tabular}\end{center}

Now, 0 gets sent to the accepting partition by $a$ and to the rejecting partition by $b$. Similarly, 2, 4, and 6 get sent to the accepting partition by $a$ and to the rejecting partition by $b$. Thus, they belong to the same partition.

In the same vein, 1 gets sent to the rejecting partition by $a$ and to the accepting partition by $b$. Similarly, 3, 5, and 7 get sent to the rejecting partition by $a$ and to the accepting partition by $b$. Thus, our next partition is

\begin{center}\begin{tabular}{|c|c|c|c|c|c|c|c|}
\multicolumn{4}{|c|}{Rejecting} & \multicolumn{4}{|c|}{Accepting}\\\hline
\multicolumn{4}{|c|}{0, 2, 4, 6} & \multicolumn{4}{|c|}{1, 3, 5, 7}\\
\multicolumn{4}{|c|}{0, 2, 4, 6} & \multicolumn{4}{|c|}{1, 3, 5, 7}
\end{tabular}\end{center}

That our row is the same as the preceding one indicates that we have finished, and now have a minimal DFA. Call the first subset $p$ and the second $q$. When an element in $p$ receives $a$, it is sent to $q$. When it receives $b$, it is sent to $p$. Similar logic for $q$ gives our new DFA:

\begin{center}\begin{tabular}{r| c c r}
         & $a$ & $b$ & \\\hline
         $\to p$ & $q$ & $p$ & 0\\
               $q$ & $p$ & $q$ & 1\\
    \end{tabular}\end{center}

Recall that $p$ began as a subset of the rejecting elements and $q$ the accepting elements, which informs the last column of the above table.
    
Not all DFAs can be reduced. An obvious example is the above reduced DFA. For a less trivial example, consider the following DFA:
   
    \begin{center}\begin{tabular}{r| c c r}
         & $a$ & $b$ & \\\hline
         $\to 0$ & 1 & 2 & 0\\
               1 & 2 & 3 & 1\\
               2 & 3 & 4 & 0\\
               3 & 0 & 5 & 1\\
               4 & 5 & 6 & 0\\
               5 & 6 & 7 & 1\\
               6 & 7 & 0 & 0\\
               7 & 4 & 2 & 1
    \end{tabular}\end{center}

Begin, as in the previous problem, with two partitions:

\begin{center}\begin{tabular}{c|c|c|c|c|c|c|c}
\multicolumn{4}{c|}{Rejecting} & \multicolumn{4}{|c}{Accepting}\\\hline
\multicolumn{4}{c|}{0, 2, 4, 6} & \multicolumn{4}{|c}{1, 3, 5, 7}
\end{tabular}\end{center}

As in the previous problem, 0, 2, 4, and 6 get sent to the same partition under $a$ and $b$, respectively. Under $a$, 1, 3, 5, and 7 go to the rejecting partition. However, under $b$, 7 goes to the rejecting partition while 1, 3, and 5 go to the accepting partition, which means we must create a new partition for 7.  

\begin{center}\begin{tabular}{|c|c|c|c|c|c|c|c|}
\multicolumn{4}{|c|}{Rejecting} & \multicolumn{4}{|c|}{Accepting}\\\hline
\multicolumn{4}{|c|}{0, 2, 4, 6} & \multicolumn{4}{|c|}{1, 3, 5, 7}\\\hline
\multicolumn{4}{|c|}{0, 2, 4, 6} & \multicolumn{3}{|c|}{1, 3, 5} & 7
\end{tabular}\end{center}

We continue the process, noting that there is no need to consider singletones, i.e., the partition $\{7\}$ is already in its finale state. Under $a$, 0, 2, and 4 get sent to the $\{1, 3, 5\}$ partition. Under $b$, they get sent to the $\{0, 2, 4, 6\}$ partition. However, 6 gets sent to the $\{7\}$ partition, and so it must be partitioned separately. Similarly, 1 and 3 get sent to the $\{0, 2, 4, 6\}$ partition under $a$, and to the $\{1, 3, 5\}$ partition under b. 5, on the other hand, gets sent to the $\{7\}$ partition, and must be partitioned separately. In total, we have:

\begin{center}\begin{tabular}{|c|c|c|c|c|c|c|c|}
\multicolumn{4}{|c|}{Rejecting} & \multicolumn{4}{|c|}{Accepting}\\\hline
\multicolumn{4}{|c|}{0, 2, 4, 6} & \multicolumn{4}{|c|}{1, 3, 5, 7}\\\hline
\multicolumn{4}{|c|}{0, 2, 4, 6} & \multicolumn{3}{|c|}{1, 3, 5} & 7\\\hline
\multicolumn{3}{|c|}{0, 2, 4} & 6 & \multicolumn{2}{|c|}{1, 3} & 5 & 7
\end{tabular}\end{center}

We continue:

\begin{center}\begin{tabular}{|c|c|c|c|c|c|c|c|}
\multicolumn{4}{|c|}{Rejecting} & \multicolumn{4}{|c|}{Accepting}\\\hline
\multicolumn{4}{|c|}{0, 2, 4, 6} & \multicolumn{4}{|c|}{1, 3, 5, 7}\\\hline
\multicolumn{4}{|c|}{0, 2, 4, 6} & \multicolumn{3}{|c|}{1, 3, 5} & 7\\\hline
\multicolumn{3}{|c|}{0, 2, 4} & 6 & \multicolumn{2}{|c|}{1, 3} & 5 & 7\\\hline
\multicolumn{2}{|c|}{0, 2} & 4 & \multicolumn{1}{c|}{6} & \multicolumn{1}{|c|}{1} & 3 & 5 & 7\\\hline
0 & 2 & 4 & \multicolumn{1}{c|}{6} & \multicolumn{1}{|c|}{1} & 3 & 5 & 7
\end{tabular}\end{center}

Notice that the reduced DFA has 8 states, like the original! This means that the original DFA is already reduced, and cannot be reduced further.

\subsubsection{Regular Expressions}
\begin{definition}
Given an alphabet $A$, we define a \textbf{regular expression}
\begin{enumerate}[(a)]
    \item \begin{itemize}
        \item $a\in A$ is a regular expression denoting the language $\{a\}$
        \item $\varepsilon$ is a regular expression denoting $\{\varepsilon\}$
        \item $\emptyset$ is a regular expression denoting $\emptyset$
    \end{itemize}
    
    \item If $\alpha$ and $\beta$ are regular expressions denoting the languages $L(\alpha)$ and $L(\beta)$, respectively, then 
    \begin{itemize}
        \item $\alpha\cup\beta$ denotes $L(\alpha)\cup L(\beta)$
        \item $\alpha\cdot\beta$ denotes $L(\alpha)\cdot L(\beta)$
        \item $\alpha^*$ denotes $L(\alpha)^*$
    \end{itemize}
\end{enumerate}
\end{definition}

By convention, we define precedence of the operations $\cup$, $\cdot$, and $^*$ in that order. Thus, \[b\cdot a^*\cup c=(b\cdot(a^*))\cup c)\]

A regular expression $\alpha$ over an alphabet $A$ denotes the set of languages which accept $\alpha$. Thus, we would like to construct an NFA $\underset{\sim}{N}$ such that $L(\underset{\sim}{N})=L(\alpha)$.

Suppose we wish to construct an NFA for only the letter $a$, i.e., this NFA rejects all strings but $a$. 

An NFA for only $\varepsilon$ would appear as:

And finally, an NFA for only $\emptyset$ is:

Now, suppose we have an NFA for $\alpha$ and $\beta$. We wish to determine NFAs for $\alpha\cup\beta$, $\alpha\cdot\beta$, and $\alpha^*$. 

We define \begin{align*}
      \underset{\sim}{N_\alpha} &= (A, Q_{\alpha}, \tau_{\alpha}, q_0)\\
      \underset{\sim}{N_\beta}  &= (A, Q_{\beta}, \tau_{\beta}, q_0)
\end{align*}

such that \begin{align*}
      L(\underset{\sim}{N_\alpha}) &= L(\alpha)\\
      L(\underset{\sim}{N_\beta})  &= L(\beta)\\
      Q_\alpha\cap Q_\beta&=\{q_0\}
\end{align*}

and clarify that these automata are non-returning, i.e., that $q_0\not\in\tau(q_0, s)$ for any $s$ of length 1 or greater.

\subsubsection{Solutions of Certain Language Equations}

Given a regular expression, we can form an NFA which admits the same language by solving \textbf{Language Equations}. We show the following lemma before proceeding to examples:

\begin{lemma}
      If $X=L\cdot X\cup M$ then $X=L^*\cdot M$ is a solution, and is unique if $\varepsilon\not\in L$. 
\end{lemma}

\begin{proof}
      Clearly, $L^*\cdot M$ is a solution, since \[L^*\cdot M = L\cdot (L^*\cdot M)\cup M\]. To prove uniqueness, suppose $s_1$ and $s_2$ are distinct solutions. There must exist a shortest-length string in $s_1$, say $s$. 
\end{proof}

Consider the following NFA:

\begin{center}\begin{tabular}{r| c c r}
      & $a$ & $b$ & \\\hline
      $\to 1$ & 2 & 1, 3 & 0\\
            2 & $\emptyset$ & 3 & 0\\
            3 & 2, 3 & 1 & 1
 \end{tabular}\end{center}

This admits the following set of equations 

\begin{align}
      X_1 &= aX_2\cup bX_1\cup bX_3\\
      X_2 &= bX_3\\
      X_3 &= aX_2\cup aX_3\cup bX_1\cup\varepsilon
\end{align}

We substitute (2) into (1) and (3):

\begin{align*}
      X_1 &= abX_3\cup bX_1\cup bX_3\\
      X_3 &= abX_3\cup aX_3\cup bX_1\cup\varepsilon
\end{align*}

which we rewrite as

\begin{align*}
      X_1 &= (ab\cup b)X_3\cup bX_1\\
      X_3 &= (ab\cup a)X_3\cup bX_1\cup\varepsilon
\end{align*}

We now apply our lemma to the equation for $X_3$

\begin{align*}
      X_1 &= (ab\cup b)X_3\cup bX_1\\
      X_3 &= (ab\cup a)^*(bX_1\cup\varepsilon)
\end{align*}

We substitute $X_3$ into the equation for $X_1$

\begin{align*}
      X_1&=(ab\cup b)(ab\cup a)^*(bX_1\cup\varepsilon)\cup bX_1\\
         &=\left((ab\cup b)(ab\cup a)^*\cup b\right)X_1\cup (ab\cup b)(ab\cup a)^*\cup bX_1\\
         &=\left((ab\cup b)(ab\cup a)^*\cup b\right)X_1\cup (ab\cup b)(ab\cup a)^*\\
         &=\left((ab\cup b)(ab\cup a)^*\cup b\right)^*(ab\cup b)(ab\cup a)^*
\end{align*}

Consider the example:

\begin{center}\begin{tabular}{r| c c r}
      & $a$ & $b$ & \\\hline
      $\to 1$ & 2 & 3 & 0\\
            2 & 2 & 3 & 0\\
            3 & 2, 3 & 1 & 1
 \end{tabular}\end{center}

 This admits the following system of equations:

 \begin{align}
      X_1 &= aX_2\cup bX_3\\
      X_2 &= aX_2\cup bX_3\\
      X_3 &= aX_2\cup bX_3\cup\varepsilon
\end{align}

From our lemma, we have $X_2=a^*bX_3$:

\begin{align}
      X_1 &= aa^*bX_3\cup bX_3\\
      X_3 &= aa^*bX_3\cup bX_3\cup\varepsilon
\end{align}

which can be simplified:

\begin{align}
      X_1 &= (aa^*b\cup b)X_3\\
      X_3 &= (aa^*b\cup b)X_3\cup\varepsilon
\end{align}

Applying our lemma to $X_3$, we have \[X_3=(aa^*b\cup b)^*\] Subsituting into $X_1$ gives \[X_1=(aa^*b\cup b)(aa^*b\cup b)^*\] 

Given any regular expression, the number of states in its corresponding NFA is the same as the number of symbols in the regular expression. 
\subsubsection{Regular Grammars}
