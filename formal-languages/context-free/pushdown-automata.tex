\subsection{Pushdown Automata}\label{subsec:pushdown-automata}
\begin{definition}
    A \textbf{Pushdown Automaton} is a septuple \[\underset{\sim}{P}=\left(Q, T, \Gamma, \delta, q_0, Z_0, F\right)\] where

    \begin{align*}
        &Q \text{ is a finite non-empty set of states}\\
        &T \text{ is an alphabet of input symbols}\\
        &\Gamma \text{ is an alphabet of stack symbols}\\
        &\delta: Q\times (T\cup \{\varepsilon \})\times\Gamma\to Q\times \Gamma^* \text{ is the move function}\\
        &q_0\in Q \text{ is the initial state}\\
        &Z_0\in\Gamma \text{ is the initial stack symbol}\\
        &F\subseteq Q\text{ is the set of accepting states}
    \end{align*}

    We define \((q, a, \gamma)\vdash(q', a', \gamma')\) when \(\delta(q, a, \gamma)=(q', a', \gamma')\).
\end{definition}

Informally, a pushdown automaton differs from a finite state machine in that it has a stack that allows it to: 

\begin{enumerate}[1.]
    \item Choose a transition based on the top of the stack
    \item Push to or pop from the stack during a transition
\end{enumerate}

\begin{definition}
    Let \(\underset{\sim}{P}=(Q, T, \Gamma, \delta, q_0, Z_0, F\)) be a pushdown automaton. The \textbf{language accepted by final state} is given by \[L(\underset{\sim}{P})=\{w\in T^*\mid (q_0, w, z_0) \vdash^* (f, \varepsilon, \gamma), \text{ where } f\in F \text{ and } \gamma\in\Gamma^*\} \]
    The \textbf{language accepted by empty stack} is given by 
    \[N(\underset{\sim}{P})=\{w\in T^*\mid (q_0, w, z_0) \vdash^* (q, \varepsilon, \varepsilon), \text{ where } q\in Q\} \]

    where \(\vdash^*\) denotes 0 or more transitions. 
\end{definition}

Consider the following example:

\begin{center}\begin{tabular}{l r c c c}
    & & \(a\) & \(b\) & \(\varepsilon \) \\\cmidrule{3-5}
    \multirow{2}{*}{\(q_0\)} & \(z_0\) & \((q_0, z_0z)\) & \(\emptyset \) & \(\emptyset \) \\
    & \(z\) & \((q_0, zz)\)  & \((q_1, \varepsilon) \) & \(\emptyset \) \\\midrule
    \multirow{2}{*}{\(q_1\)} & \(z_0\) & \(\emptyset \) & \(\emptyset \) & \((q_f, \varepsilon)\) \\
    & \(z\) & \(\emptyset \) & \((q_1, \varepsilon)\) & \(\emptyset \) \\\midrule
    \multirow{2}{*}{\(q_f\)} & \(z_0\) & \multicolumn{3}{c}{\multirow{2}{*}{accept}}\\
    & \(z\) & & &\\
\end{tabular}\end{center}



This describes the language \(L=\{a^n b^n\mid n\geq1\} \)

Consider the following language on \( \{0, 1\} \):

\[L=\{w2w^R \mid w\in {\{0, 1\}}^*\} \]

where \(w^R\) is the reflection of \(w\). This is given by the context-free grammar 

\[S\to 0S0\mid 1S1\mid 2\]

This can also be defined by the following pushdown automaton:

\begin{center}\begin{tabular}{l r c c c c}
    & & 0 & 1 & 2 & \(\varepsilon \) \\\cmidrule{3-6}
    \multirow{2}{*}{\(q_0\)} & \(z_0\) & \((q_0, z_0z)\) & \(\emptyset \) & \(\emptyset \) \\
    & \(z\) & \((q_0, zz)\)  & \((q_1, \varepsilon) \) & \(\emptyset \) \\\midrule
    \multirow{2}{*}{\(q_1\)} & \(z_0\) & \(\emptyset \) & \(\emptyset \) & \((q_f, \varepsilon)\) \\
    & \(z\) & \(\emptyset \) & \((q_1, \varepsilon)\) & \(\emptyset \) \\\midrule
    \multirow{2}{*}{\(q_f\)} & \(z_0\) & \multicolumn{3}{c}{\multirow{2}{*}{accept}}\\
    & \(z\) & & &\\
\end{tabular}\end{center}

\begin{theorem}
    For any pushdown automaton \(M\), there exist pushdown automata \(P\) and \(Q\) such that \(L(M)=N(P)\) and \(N(L)=L(Q)\).
\end{theorem}