\subsection{Context-Free Grammars}\label{subsec:context-free-grammars}
\begin{definition}
    A \textbf{Context-Free Grammar} is a quartuple \(G=(N, T, P, S)\) where 
          \begin{align*}
                & N\text{ is a finite, non-empty set of variables (also called non-terminals)}\\
                & T\text{ is an alphabet of terminals}\\
                & P\subseteq N\times{(N\cup T)}^*\text{ is a finite set of productions}\\
                & S\in N\text{ is the starting symbol}
          \end{align*}

          For any \((A, \gamma)\in P\), we write \(A\to\gamma \), and say \(A\) \textbf{produces} \(\gamma \). 
\end{definition}

By convention, we use upper-case letters to denote variables, lower-case to denote terminals and strings over the terminals, and Greek letters to denote strings over variables and terminals.

\begin{definition}
    Given strings \(\alpha \) and \(\beta \), we say \(\alpha \) \textbf{derives} \(\beta \) if there exist \(A, \alpha_1,\alpha_2,\gamma \) such that 
    \begin{align*}
          \alpha &= \alpha_1A\alpha_2\\
          \beta  &= \alpha_1\gamma\alpha_2\\
          A&\to \gamma\in P
    \end{align*}

    and we write this \(\alpha\Rightarrow\beta \). 
\end{definition}

We can define the language of a context-free grammar:

\begin{definition}
      Given a context-free grammar \(G\), the corresponding \textbf{context-free language} is \[L(G)=\{w \mid S\Rightarrow w\} \]
\end{definition}

\begin{theorem}
      Every regular language is a context-free language. 
\end{theorem}

\begin{proof}
      Let \(L\) be a regular language and let \(N=(Q, A, \tau, q_0, \mathcal{F})\) be its corresponding minimum DFA\@. 
\end{proof}

However, not every context-free language is regular. For example, the language \( \{a^n b^n \mid n\geq0\} \) is not regular, but is a context-free language given by