\subsubsection{Regular Expressions}\label{subsubsec:regular-expressions}
\begin{definition}
Given an alphabet \(A\), we define a \textbf{regular expression}
\begin{enumerate}[(a)]
    \item \begin{itemize}
        \item \(a\in A\) is a regular expression denoting the language \( \{a\} \)
        \item \(\varepsilon \) is a regular expression denoting \( \{\varepsilon \} \)
        \item \(\emptyset \) is a regular expression denoting \(\emptyset \)
    \end{itemize}
    
    \item If \(\alpha \) and \(\beta \) are regular expressions denoting the languages \(L(\alpha)\) and \(L(\beta)\), respectively, then 
    \begin{itemize}
        \item \(\alpha\cup\beta \) denotes \(L(\alpha)\cup L(\beta)\)
        \item \(\alpha\cdot\beta \) denotes \(L(\alpha)\cdot L(\beta)\)
        \item \(\alpha^*\) denotes \(L{(\alpha)}^*\)
    \end{itemize}
\end{enumerate}
\end{definition}

By convention, we define precedence of the operations \(\cup \), \(\cdot \), and \(^*\) in that order. Thus, \[b\cdot a^*\cup c=(b\cdot(a^*))\cup c\]

A regular expression \(\alpha \) over an alphabet \(A\) denotes the set of languages which accept \(\alpha \). Thus, we would like to construct an NFA \(\underset{\sim}{N}\) such that \(L(\underset{\sim}{N})=L(\alpha)\).

The following NFA rejects all strings but \(a\):

\begin{center}\begin{tabular}{r c c r}
      & \(a\) & \(b\neq a\) & \\\bottomrule
      \(\to q_0\) & \(q_1\) & \(\emptyset \) & 0\\
            \(q_1\) & \(\emptyset \) & \(\emptyset \) & 1\\
 \end{tabular}\end{center}

An NFA for only \(\varepsilon \) would appear as:

\begin{center}\begin{tabular}{r c r}
      & \(c\in A\) & \\\bottomrule
      \(\to q_0\) & \(\emptyset \) & 1\\
 \end{tabular}\end{center}

And finally, an NFA for only \(\emptyset \) is:

\begin{center}\begin{tabular}{r c r}
      & \(c\in A\) & \\\bottomrule
      \(\to q_0\) & \(\emptyset \) & 0\\
 \end{tabular}\end{center}

Now, suppose we have an NFA for \(\alpha \) and \(\beta \). We wish to determine NFAs for \(\alpha\cup\beta \), \(\alpha\cdot\beta \), and \(\alpha^*\). 

We define \begin{align*}
      \underset{\sim}{N_\alpha} &= (A, Q_{\alpha}, \tau_{\alpha}, q_0, \mathcal{F}_{\alpha})\\
      \underset{\sim}{N_\beta}  &= (A, Q_{\beta}, \tau_{\beta}, q_0, \mathcal{F}_{\beta})
\end{align*}

such that \begin{align*}
      L(\underset{\sim}{N_\alpha}) &= L(\alpha)\\
      L(\underset{\sim}{N_\beta})  &= L(\beta)\\
      Q_\alpha\cap Q_\beta&=\{q_0\}
\end{align*}

and clarify that these automata are non-returning, i.e., that \(q_0\not\in\tau(q_0, s)\) for any \(s\) of length 1 or greater.

We construct the \textbf{Union} \[\underset{\sim}{N_{\alpha\cup\beta}}=(A, Q_{\alpha\cup\beta}, \tau_{\alpha\cup\beta}, q_0, \mathcal{F}_{\alpha\cup\beta})\] 

where \(Q_{\alpha\cup\beta}=Q_{\alpha}\cup Q_{\beta}\), \(\mathcal{F}_{\alpha\cup\beta}=\mathcal{F}_{\alpha}\cup F_{\beta}\) and, for all \(q\in Q_{\alpha\cup\beta}\) and \(a\in A\)

\[\tau_{\alpha\cup\beta}(q, a) = \begin{cases}
      \tau_{\alpha}(q_0, a)\cup\tau_{\beta}(q_0, a) & \text{ if } q=q_0\\
      \tau_{\alpha}(q, a) & \text{ if } q\in Q_{\alpha}-\{q_0\} \\
      \tau_{\beta}(q, a) & \text{ if } q\in Q_{\beta}-\{q_0\} 
\end{cases}\]

The \textbf{Concatenation} is constructed 

\[\underset{\sim}{N_{\alpha\beta}}=(A, Q_{\alpha\beta}, \tau_{\alpha\beta}, q_0, \mathcal{F}_{\alpha\beta})\] 

where \(Q_{\alpha\beta}=Q_{\alpha}\cup Q_{\beta}\),

\[\mathcal{F}_{\alpha\beta}=\begin{cases}\mathcal{F}_{\beta} &\text{ if } q_0\not\in \mathcal{F}_{\beta}\\ \mathcal{F}_{\alpha}\cup(\mathcal{F}_{\beta}-\{q_0\}) &\text{ if } q_0\in \mathcal{F}_{\beta}\end{cases}\] and, for all \(q\in Q_{\alpha\beta}\) and \(a\in A\)

\[\tau_{\alpha\beta}(q, a) = \begin{cases}
      \tau_{\alpha}(q, a)\cup\tau_{\beta}(q_0, a) & \text{ if } q\in\mathcal{F}_{\alpha}\\
      \tau_{\alpha}(q, a) & \text{ if } q\in Q_{\alpha}-\mathcal{F}_{\alpha}\\
      \tau_{\beta}(q, a) & \text{ if } q\in Q_{\beta}-\{q_0\} 
\end{cases}\]

Finally, the \textbf{Kleene Closure} is constructed 

\[\underset{\sim}{N_{\alpha^*}}=(A, Q_{\alpha^*}, \tau_{\alpha^*}, q_0, \mathcal{F}_{\alpha^*})\] 

where \(Q_{\alpha^*}=Q_{\alpha}\), \(\mathcal{F}_{\alpha^*}=\mathcal{F}_{\alpha}\cup \{q_0\} \) and, for all \(q\in Q_{\alpha^*}\) and \(a\in A\)

\[\tau_{\alpha^*}(q, a) = \begin{cases}
      \tau_{\alpha}(q, a)\cup\tau_{\alpha}(q_0, a) & \text{ if } q\in\mathcal{F}_{\alpha}\\
      \tau_{\alpha}(q, a) & \text{ if } q\in Q_{\alpha}-\mathcal{F}_{\alpha}
\end{cases}\]

This allows us to construct NFAs from a regular expression. Suppose we have a regular expression \(ab\) over \( \{a, b\} \). Then we have 
\begin{center}
\begin{multicols}{2}
\begin{tabular}{r c c r}
      \multicolumn{4}{c}{NFA for \(a\)}\\
      & \(a\) & \(b\) & \\\bottomrule
      \(\to q_0\) & \(q_1\) & \(\emptyset \) & 0\\
            \(q_1\) & \(\emptyset \) & \(\emptyset \) & 1
 \end{tabular}

 \begin{tabular}{r c c r}
      \multicolumn{4}{c}{NFA for \(b\)}\\
      & \(a\) & \(b\) & \\\bottomrule
      \(\to q_0\) & \(\emptyset \) & \(q_2\) & 0\\
            \(q_2\) & \(\emptyset \) & \(\emptyset \) & 1
 \end{tabular}
\end{multicols}
\end{center}

Applying the above construction for concatenation gives
\begin{center}
\begin{tabular}{r c c r}
      \multicolumn{4}{c}{NFA for \(ab\)}\\
      & \(a\) & \(b\) & \\\bottomrule
      \(\to q_0\) & \(q_1\) & \(\emptyset \) & 0\\
          \(q_1\) & \(\emptyset \) & \(q_2\) & 0\\
          \(q_2\) & \(\emptyset \) & \(\emptyset \) & 1
 \end{tabular}
\end{center}
