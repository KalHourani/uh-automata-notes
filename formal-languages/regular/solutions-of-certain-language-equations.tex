\subsubsection{Solutions of Certain Language Equations}\label{subsubsec:solutions-of-certain-language-equations}

Given a regular expression, we can form an NFA which admits the same language by solving \textbf{Language Equations}. We show the following lemma before proceeding to examples:

\begin{lemma}
      If \(X=L\cdot X\cup M\) then \(X=L^*\cdot M\) is a solution, and is unique if \(\varepsilon\not\in L\). 
\end{lemma}

\begin{proof}
      Clearly, \(L^*\cdot M\) is a solution, since \[L^*\cdot M = L\cdot (L^*\cdot M)\cup M\] To prove uniqueness, suppose \(s_1\) and \(s_2\) are distinct solutions. There must exist a shortest-length string in \(s_1\), say \(s\). 
\end{proof}

Consider the following NFA\@:

\begin{center}\begin{tabular}{r c c r}
      & \(a\) & \(b\) & \\\bottomrule
      \(\to 1\) & 2 & 1, 3 & 0\\
            2 & \(\emptyset \) & 3 & 0\\
            3 & 2, 3 & 1 & 1
 \end{tabular}\end{center}

This admits the following set of equations 

\begin{align}
      X_1 &= aX_2\cup bX_1\cup bX_3\\
      X_2 &= bX_3\\
      X_3 &= aX_2\cup aX_3\cup bX_1\cup\varepsilon
\end{align}

We substitute (2) into (1) and (3):

\begin{align*}
      X_1 &= abX_3\cup bX_1\cup bX_3\\
      X_3 &= abX_3\cup aX_3\cup bX_1\cup\varepsilon
\end{align*}

which we rewrite as

\begin{align*}
      X_1 &= (ab\cup b)X_3\cup bX_1\\
      X_3 &= (ab\cup a)X_3\cup bX_1\cup\varepsilon
\end{align*}

We now apply our lemma to the equation for \(X_3\)

\begin{align*}
      X_1 &= (ab\cup b)X_3\cup bX_1\\
      X_3 &= {(ab\cup a)}^*(bX_1\cup\varepsilon)
\end{align*}

We substitute \(X_3\) into the equation for \(X_1\)

\begin{align*}
      X_1 &= (ab\cup b){(ab\cup a)}^*(bX_1\cup\varepsilon)\cup bX_1\\
          &= \left((ab\cup b){(ab\cup a)}^*\cup b\right)X_1\cup (ab\cup b){(ab\cup a)}^*\cup bX_1\\
          &= \left((ab\cup b){(ab\cup a)}^*\cup b\right)X_1\cup (ab\cup b){(ab\cup a)}^*\\
          &= {\left((ab\cup b){(ab\cup a)}^*\cup b\right)}^*(ab\cup b){(ab\cup a)}^*
\end{align*}

Consider the example:

\begin{center}\begin{tabular}{r c c r}
      & \(a\) & \(b\) & \\\bottomrule
      \(\to 1\) & 2 & 3 & 0\\
            2 & 2 & 3 & 0\\
            3 & 2 & 3 & 1
 \end{tabular}\end{center}

 This admits the following system of equations:

 \begin{align*}
      X_1 &= aX_2\cup bX_3\\
      X_2 &= aX_2\cup bX_3\\
      X_3 &= aX_2\cup bX_3\cup\varepsilon
\end{align*}

From our lemma, we have \(X_2=a^*bX_3\):

\begin{align*}
      X_1 &= aa^*bX_3\cup bX_3\\
      X_3 &= aa^*bX_3\cup bX_3\cup\varepsilon
\end{align*}

which can be simplified:

\begin{align*}
      X_1 &= (aa^*b\cup b)X_3\\
      X_3 &= (aa^*b\cup b)X_3\cup\varepsilon
\end{align*}

Applying our lemma to \(X_3\), we have \[X_3={(aa^*b\cup b)}^*\] Subsituting into \(X_1\) gives \[X_1=(aa^*b\cup b){(aa^*b\cup b)}^*\] 

One final example:

\begin{center}\begin{tabular}{r c c r}
      & \(a\) & \(b\) & \\\bottomrule
      \(\to 1\) & \(\emptyset \) & 1, 2 & 1\\
            2 & 1 & \(\emptyset \) & 1\\
 \end{tabular}\end{center}

\begin{align*}
      X_1 &= bX_1\cup bX_2\cup\varepsilon \\
      X_2 &= aX_1\cup\varepsilon
\end{align*}

Substituting our equation for \(X_2\) into \(X_1\) gives 

\begin{align*}
      X_1&=bX_1\cup b(aX_1\cup\varepsilon)\cup\varepsilon \\
         &=(b\cup ba)X_1\cup b\cup\varepsilon \\
         &={(b\cup ba)}^*(b\cup\varepsilon)
\end{align*}

