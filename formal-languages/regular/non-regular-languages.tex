\subsubsection{Non-Regular Languages}\label{subsubsec:non-regular-languages}
Suppose we have a DFA \[\underset{\sim}{D}=(A, Q, \tau, q_0, \mathcal{F})\] with \(|Q|=n\). Consider the following set of equations

\begin{align*}
      q_1 &= \tau(q_0, a_1)\\
      q_2 &= \tau(q_1, a_2)\\
          &\vdots\\
      q_i &= \tau(q_{i-1}, a_i)\\
          &\vdots\\
      q_{n-1} &= \tau(q_{n-2}, a_{n-1})\\
      q_n &= \tau(q_{n-1}, a_n)
\end{align*}

Notice that we have \(n+1\) states above, but only \(n\) states in \(Q\). Then, by the Pidgeonhole Principle, there must be some state \(q_i\) that is visited twice. In other words, there exist an \(i, j\) with \(i < j\) such that \(q_i=q_j\). We then consider a string \(s=a_1a_2\hdots a_n\). Let \(v_1=a_1a_2\hdots a_i\), \(v_2=a_{i+1}a_{i+2}\hdots a_j\), and \(v_3=a_{j+1}a_{j+2}\hdots a_n\). We see that 

\[\tau(q_0, s)=\tau(q_0, v_1v_2^kv_3)\] for all \(k\geq0\). 

\begin{center}\begin{tikzpicture}[shorten >=1pt,node distance=2cm,on grid,auto]
      \tikzstyle{every state}=[fill={rgb:black,1;white,10}]
    
      \node[state,initial](q_0){\(q_0\)};
      \node[state](q_i)[right of=q_0]{\(q_i\)};
      \node[state](q_n)[right of=q_i]{\(q_n\)};    
      \path[->]
      (q_0) edge                node {\(v_1\)}  (q_i)
      (q_i) edge                node {\(v_3\)}  (q_n)
            edge  [loop above]  node {\(v_2\)}  ();
    \end{tikzpicture}\end{center}

Thus, we state \textbf{The Pumping Lemma} and provide a proof:

\begin{theorem}[The Pumping Lemma]
      Let \(L\) be any regular language. Then there exists a \(p>0\) (called the \textbf{pumping length} such that, for any string \(s\in A^*\) such that \(|s|>=p\), we can write \[s=s_1s_2s_3\] where 
      \begin{itemize}
            \item \(|s_2|\geq 1\)
            \item \(|s_1s_2|\leq p\)
            \item \(L\) accepts \(s_1s_2^ns_3\) for all \(n\geq0\).
      \end{itemize}
\end{theorem}

\begin{proof}

\end{proof}

We can use the above theorem to prove that certain languages are not regular. 
