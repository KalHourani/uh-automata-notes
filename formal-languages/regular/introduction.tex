\subsection{Introduction}\label{subsubsec:introduction}
\begin{definition}
 An \textbf{Alphabet} is a finite, non-empty set of atomic symbols.
\end{definition}

\begin{definition}
 A \textbf{word} or \textbf{string} is any finite sequence of symbols from an alphabet. 
\end{definition}

\begin{definition}
 The \textbf{length} of a string, \(s\), denoted \(\abs{s}\), is the number of symbols in \(s\). 
\end{definition}

\begin{definition}
 Given strings \(s=s_1s_2\hdots s_n\) and \(t=t_1t_2\hdots t_m\), their \textbf{concatenation} is defined 
 
 \[s\cdot t=s_1s_2\cdots s_n t_1t_2\cdots t_m\]
\end{definition}

We denote by \(\varepsilon \) the \textbf{empty string}, the unique string of 0 characters. 

\begin{definition}
 Let \(A\) be any alphabet. The \textbf{Kleene Closure} of \(A\), denoted \(A^*\), is the set of all strings of any length over \(A\).  
\end{definition}

\begin{theorem}
 Let \(A\) be any finite set. Then \(A^*\) is countably infinite. 
\end{theorem}

\begin{proof}
 That \(A^*\) is infinite is straightforward: since \(A\) is non-empty, take \(a\in A\). Then \[ \{a, aa, aaa, \hdots \}\subseteq A^*\]
 
 To see that it is countable, we first write \(\abs{A}=n\). Now, consider the set of all strings of length 0. This is simply \( \{\varepsilon \} \). Moreover, there are \(n\) strings of length 1, \(n^2\) strings of length 2, \(n^3\) strings of length 3, and so on. Thus, we map \(\varepsilon \) to 0, the strings of length 1 to \(1, 2, \hdots, n\), the strings of length 2 to \(n+1, n+2, \hdots, n+n^2\), the strings of length 3 to \(n+n^2+1, n+n^2+2,\hdots, n+n^2+n^3\), and so on. This is a bijection from \(A^*\) to \(\mathbb{N}\), which completes the proof.
\end{proof}

\begin{definition}
 Given an alphabet \(A\), a \textbf{formal language} or simply \textbf{language} \(L\) is any subset of \(A^*\). 
\end{definition}

\begin{theorem}
 Given an alphabet \(A\), the set of languages over \(A\) is uncountable. 
\end{theorem}

\begin{proof}
 Suppose, by way of contradiction, that the set of languages were countable, i.e., that we can enumerate the set as \( \{L_1, L_2, L_3, \hdots \} \). Consider the set of all strings \( \{s_1, s_2, s_3, \hdots \} \). Let \(L\) be the language defined as follows: 
 
\[s_i\in L\text{ if and only if } s_i\not\in L_i\]

 To see that \(L\) is not in the above list, consider \(s_i\). If \(s_i\) is in \(L\), then \(s_i\) is not in \(L_i\), by construction, and \(L\neq L_i\). Similarly, if \(s_i\) is not in \(L\), then \(s_i\) must be in \(L_i\), by construction, and \(L\neq L_i\). In other words, for all \(i\), \(L\neq L_i\). Then \(L\) is not in the above list, which is a contradiction. Hence, the set of languages is uncountable. 
\end{proof}

All set operations, such as union, intersection, complement, set-difference, etc.\@ can be applied to languages, since languages are simply subsets of a Kleene Closure of an alphabet.

\begin{definition}
 Given two languages \(L_1\) and \(L_2\), the concatenation \(L_1\cdot L_2\) is given by
 
 \[L_1\cdot L_2=\{s\cdot t \mid s\in L_1\text{ and } t\in L_2\} \]
\end{definition}

Clearly, we have 

\begin{align*}
    L\cdot\emptyset &= \emptyset = \emptyset\cdot L\\
    L\cdot \{\varepsilon \} &= L = \{\varepsilon \}\cdot L
\end{align*}

Note that \(L_1\cdot L_2\) is not the same as \(L_1\times L_2\). Let \(L_1=L_2=\{\varepsilon, 0, 00\} \). Then

\[L_1\times L_2=\{(\varepsilon, \varepsilon), (\varepsilon, 0), (\varepsilon, 00), (0, \varepsilon), (0, 0), (0, 00), (00, \varepsilon), (00, 0), (00, 00)\} \]

whereas 

\[L_1\cdot L_2=\{\varepsilon, 0, 00, 000, 0000\} \]

\begin{definition}
 Given a language \(L\), the \textbf{Kleene Closure} of \(L\), \(L^*\), is 
 
 \[L^*=\bigcup_{i=0}^{\infty}L^i\]
 
 where \[L^i = \begin{cases} \{\varepsilon \} &\mbox{if } i = 0 \\
L \cdot L^{i-1} & \mbox{otherwise }\end{cases}\]
\end{definition}

Note that, while \(0^0\) is normally left undefined, we define \(\emptyset^0=\{\varepsilon \} \).

\begin{theorem}
 \(L^*\) is finite if and only if \(L=\emptyset \) or \(L=\{\varepsilon \} \).
\end{theorem}

\begin{proof}
 If \(L=\emptyset \), then \(L^i=\emptyset^i=\emptyset \) for \(i>0\). Then 
 
 \begin{align*}\emptyset^*&=\bigcup_{i=0}^{\infty}\emptyset^i\\
                          &=\emptyset^0\cup\bigcup_{i=1}^{\infty}\emptyset^i\\
                          &=\{\varepsilon \}\cup\bigcup_{i=1}^{\infty}\emptyset \\
                          &=\{\varepsilon \}
\end{align*}

Similarly, if \(L=\{\varepsilon \} \), then \(L^i=\{\varepsilon \} \) for all \(i\), and 

 \begin{align*}{\{\varepsilon \}}^*&=\bigcup_{i=0}^{\infty}{\{\varepsilon \}}^i\\
                          &=\bigcup_{i=1}^{\infty}\{\varepsilon \} \\
                          &=\{\varepsilon \}
\end{align*}

However, if \(L\) is neither \(\emptyset \) nor \( \{\varepsilon \} \), then there exists a string \(s\in L\) with length at least 1. Then \(s, ss, sss, \hdots \), are in \(L^*\), hence \(L^*\) is infinite.
\end{proof}
